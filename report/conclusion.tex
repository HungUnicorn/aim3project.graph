The project's goals were to implement algorithms to measure the indegree, outdegree, PageRank, closeness and betweenness centrality of nodes within a graph with the  help of Apache Flink. Further, these algorithms were intended to run on the data set provided by the Web Data Commons project. A comprehensive analysis of the structure of the web graph should be conducted. Due to missing computation power, it was not possible to fully achieve this goal.

The goals of this project have been partly achieved. The implementation of the algorithms in order to compute the different measures have been fully completed. An evaluation of these algorithms showed that the implementations produce correct results, including degree, connectivity, PageRank, closeness, except that betweenness does not match Gephi's betweenness computation. For this evaluation an example data set was used to rapidly test the algorithms, and compare the results of our algorithms with Gephi's result. 

Due to lack of computation power, it was firstly tried to run the computations locally, and reduced the granularity from the Subdomain/Host Graph to the PLD graph. Then, a 4-node cluster was used to run the computations. The 4-node cluster completes the computations of the degree even with Subdomain/Host Graph although this cluster cannot run other computations even with PLD graph. A comprehensive network analysis of the Web Graph was only limited feasible. However, results of degree could be obtained from the computation. These information were gained previously in other projects and those results could be partly confirmed by our results. The distribution of the degree could be found as not a power-law due to the noisy tails with residual analysis in regression. The Top-10 indegree and outdegree results are consistent with previous findings.

The developed algorithms implemented with the help of the Java API of Apache Flink can be used to retrieve the measures in other networks. Additional optimizations of the implementations can be achieved. The computations can be also run on a bigger cluster to gain other results such as closeness and PageRank. 