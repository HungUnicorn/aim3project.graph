The goals of this project have been partly achieved. The implementation of the algorithms in order to compute the different measures have been fully completed. An evaluation of these algorithms showed that the implementations produce correct results, including degree, connectivity, PageRank, closeness, except that betweenness doesn't match Gephi's betweenness computation. For this evaluation an example data set was used to rapidly test the algorithms, and compare the results of our algorithms with Gephi's result. Due to lack of computation power, we first tried to run the computations locally, and reduced the granularity from the Subdomain/Host Graph to the PLD graph. Then, we use the 4-node cluster to run the computations. 4-node cluster completes the computations of degree even with subdomain/host graph although this cluster cannot run other computations even with PLD graph. A comprehensive network analysis of the Web Graph was only limited feasible. However, results of degree could be obtained from the computation. These information were gained previously in other projects and those results could be partly confirmed by our results. We find the distribution of degree is not a power-law due to the noisy tailes with residual analysis in regression, and top 10 indegree and outdegree websites are consistent with previous findings.

The developed algorithms implemented with the help of the Java API of Apache Flink can be used to retrieve the measures in other networks. Additional optimizations of the implementations can be achieved. The computations can be also run on a bigger cluster to gain other results such as closeness and PageRank. 
