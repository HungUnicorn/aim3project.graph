In the last years a vast amount of data has been collected and has been made available to the public in order for everyone to analyze this data and gain knowledge about its structure. One example for such data set is the Hyperlink Graph of the University of Mannheim which has been extracted from the Common Crawl. There are currently two versions of the graph namely the 2012 and 2014 version each covering billions of pages and hyperlinks between those pages. Analyzing this data set may be beneficially for multiple research fields like  search algorithms, SPAM detection or graph analysis algorithms.  
 
Our target is to implement the algorithms to compute the statistics of importance such as indegree and outdegree distribution, the PageRank, the closeness and the betweenness centrality. Those statistics rank the nodes from different aspects. Further, we want to analyze the data set with the help of Apache Flink to process huge parts of the data set. Based on the results statements can be made about the structure of the Hyperlink Graph and therefore  statements can be formulated about the structure of the World-Wide-Web.